\usepackage{amsmath}
\usepackage{amsthm}
\usepackage{amssymb}

\usepackage{enumitem}
\setlist{nosep}

\usepackage{mathtools}
\usepackage{physics}
\usepackage[a4paper, margin=1cm, landscape, bottom=2cm]{geometry}
\usepackage{float}
\usepackage{booktabs}
\usepackage{tabularx}
\newcolumntype{Y}{>{\centering\arraybackslash}X}

% Quotes
\usepackage[autostyle, style=english]{csquotes}

% Links
\usepackage[hidelinks, breaklinks]{hyperref}

% Captions
\usepackage[margin=1.5cm]{caption}

% Multiple columns
\usepackage{multicol}

% Font
\usepackage[T1]{fontenc}
\usepackage{newpxtext}

\usepackage{titlesec}
\titlespacing\section{0pt}{12pt plus 4pt minus 2pt}{0pt plus 2pt minus 2pt}
\titlespacing\subsection{0pt}{12pt plus 4pt minus 2pt}{0pt plus 2pt minus 2pt}
\titlespacing\subsubsection{0pt}{12pt plus 4pt minus 2pt}{0pt plus 2pt minus 2pt}
\titleformat{\section}{\normalfont\fontsize{13}{13}\bfseries\MakeUppercase}{\thesection}{1em}{}
\titleformat{\subsection}{\normalfont\fontsize{12}{12}\bfseries}{\thesubsection}{1em}{}
\titleformat{\subsubsection}{\normalfont\fontsize{11}{11}\itshape}{\thesubsubsection}{1em}{}

\usepackage[nodisplayskipstretch]{setspace}
\setstretch{0.9}
\setlength{\baselineskip}{6pt}

% Misc
\usepackage{framed}
\usepackage{booktabs}

\setlength{\parskip}{\baselineskip}%
\setlength{\parindent}{0pt}%

\DeclarePairedDelimiterX\set[1]\lbrace\rbrace{\def\given{\;\delimsize\vert\;}#1}
\newcommand{\given}{\mid}

\DeclarePairedDelimiterX\VSBars[1]\lvert\rvert{#1}
\DeclarePairedDelimiterX\VDBars[1]{\lvert\lvert}{\rvert\rvert}{#1}

\newcommand{\Real}{\mathbb{R}}
\newcommand{\Rational}{\mathbb{Q}}
\newcommand{\Complex}{\mathbb{C}}
\newcommand{\Int}{\mathbb{Z}}
\newcommand{\PosInt}{\mathbb{Z}^{+}}
\newcommand{\NegInt}{\mathbb{Z}^{-}}
\newcommand{\NatNum}{\mathbb{N}}
\newcommand{\PosReal}{\mathbb{R}^{+}}
\newcommand{\NegReal}{\mathbb{R}^{-}}

\newcommand{\AlephZero}{\aleph_{0}}

\DeclareMathOperator{\Exists}{\exists}
\DeclareMathOperator{\Forall}{\forall}

\DeclareMathOperator{\Diff}{\backslash}

\DeclarePairedDelimiter{\Ceil}{\lceil}{\rceil}
\DeclarePairedDelimiter{\Floor}{\lfloor}{\rfloor}

\newcommand{\AbsComplement}[1]{{#1}^{c}}
\newcommand{\xor}{\oplus}

\newcommand{\True}{\mathrm{T}}
\newcommand{\False}{\mathrm{F}}

\newcommand{\Domain}[1]{\operatorname{domain}({#1})}
\newcommand{\Image}[1]{\operatorname{image}({#1})}

\newcommand{\Inverse}[1]{{#1}^{-1}}

\newcommand{\Permutate}[1]{\sigma({#1})}

\DeclareMathOperator{\Sign}{sgn}

\newcommand{\Mod}[1]{\ (\mathrm{mod}\ #1)}

\newcommand{\Encrypt}[3]{\mathrm{encrypt}_{#1, #2}(#3)}
\newcommand{\Decrypt}[3]{\mathrm{decrypt}_{#1, #2}(#3)}

\newcommand{\MatrixClass}[2]{\mathcal{M}(#1, #2)}

\newcommand{\rddots}{\reflectbox{$\ddots$}}

\newenvironment{detmatrix}[1]{%
  \left\vert\begin{array}{@{}*{#1}{c}@{}}
}{%
  \end{array}\right\vert
}

\newenvironment{augmatrix}[1]{%
  \left[\begin{array}{@{}*{#1}{c}|c@{}}
}{%
  \end{array}\right]
}

\newcommand{\Prob}[1]{\mathrm{P} \left( #1 \right)}
\newcommand{\ProbS}[1]{\mathrm{P} \left[ #1 \right]}
\newcommand{\ExpVal}[1]{\mathrm{E} \left( #1 \right)}
\newcommand{\ExpValS}[1]{\mathrm{E} \left[ #1 \right]}
\newcommand{\Var}[1]{\mathrm{Var} \left( #1 \right)}

\theoremstyle{definition}
\newtheorem{theorem}{Theorem}[section]
\newtheorem{corollary}{Corollary}[theorem]
\newtheorem{lemma}[theorem]{Lemma}

\theoremstyle{remark}
\newtheorem*{remark}{Remark}
\newtheorem*{example}{Example}

\theoremstyle{definition}
\newtheorem{proposition}{Proposition}[section]

\theoremstyle{definition}
\newtheorem{definition}{Definition}[section]

\renewcommand\qedsymbol{$\blacksquare$}

\newcommand{\ApproxDistributed}{\mathrel{\dot\sim}}

\newcommand{\Bernoulli}[1]{\mathrm{Bernoulli}(#1)}
\newcommand{\Binomial}[2]{\mathrm{Binomial}(#1, #2)}
\newcommand{\Geometric}[1]{\mathrm{Geometric}(#1)}
\newcommand{\Poisson}[1]{\mathrm{Poisson}(#1)}
\newcommand{\Uniform}[2]{\mathrm{U}(#1, #2)}
